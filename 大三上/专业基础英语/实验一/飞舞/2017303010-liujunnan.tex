\documentclass[journal]{IEEEtran}
\title{Edge Intelligence (EI)-Enabled HTTP Anomaly\\
Detection Framework for the Internet of Things\\
(IoT)}
\author{Yufei An, F. Richard Yu, Fellow, IEEE, Jianqiang Li, Jianyong Chen, and Victor C.M. Leung, Fellow, IEEE}
\usepackage{amsmath}
\begin{document}
\maketitle
\begin{abstract}
Abstract—In recent years, with the rapid development of the
Internet of Things (IoT), various applications based on IoT has
become more and more popular in industrial and living sectors.
\end{abstract}
\begin{IEEEkeywords}
indent Index Terms—Edge intelligence, Internet of things (IoT),
anomaly detection, HTTP
\end{IEEEkeywords}
\section{Introduction}
\IEEEPARstart{}\quad Incredible developments in the routine use of network services and electronic applications have led to massive advances
in communications networks and the emergence of the concept
of the Internet of Things (IoT).
\\ \\ \indent \footnotesize{  Manuscript received xx xx, 2020; revised xx xx, 2020; accepted xx xx,
2020. Date of publication xx xx, 2020; date of current version xx xx,
2020.}
\section{Related Works}
\IEEEPARstart{}\quad \normalsize In this section, some related works about anomaly detection
for HTTP traffic and edge intelligence are introduced.
\subsection{Detection Methods}
In the early studies, feature matching methods are mostly
used to detect abnormal HTTP traffic.
\subsubsection{Statistical Methods: Statistical methods detect anomalies mainly dependent on the predefined threshold, mean and
standard deviation, and probabilities [11].}

\begin{gather}
\begin{align}
CF_1+CF_2=(N_1&+N_2,\overrightarrow{LS_1}+\overrightarrow{LS_2},SS_1+SS_2) \\
           D_0&=|X_i-\frac{\sum^n_{i=1} \overrightarrow{X_i}}{N}|
\end{align}
\end{gather}
\begin{thebibliography}{99}
\bibitem{ref1}D. Wang, W. Zhang, B. Song, X. Du, and M. Guizani, “Market-based
model in CR-IoT: A Q-probabilistic multi-agent reinforcement learning
approach,” IEEE Trans. Cognitive Comm. Networking, vol. 6, no. 1, pp.
179–188, Mar. 2020.
\end{thebibliography}
\end{document}